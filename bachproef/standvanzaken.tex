\chapter{\IfLanguageName{dutch}{Stand van zaken}{State of the art}}%
\label{ch:stand-van-zaken}

% Tip: Begin elk hoofdstuk met een paragraaf inleiding die beschrijft hoe
% dit hoofdstuk past binnen het geheel van de bachelorproef. Geef in het
% bijzonder aan wat de link is met het vorige en volgende hoofdstuk.

% Pas na deze inleidende paragraaf komt de eerste sectiehoofding.
VR toestellen zijn tegenwoordig te vinden in vele soorten een maten. Zo zijn er headsets gemaakt voor PC of consoles en anderen voor smartphones. Dankzij software en hardware verbeteringen bestaan er nu zelfs headsets die autonoom werken. Dit zorgt ervoor dat kabels en complexe set-ups niet nodig zijn. Deze verbeteringen zorgen ook dat de digitale wereld realistischer doet aanvoelen.

Om het overschakelen naar een volgend fragment te laten gebeuren zonder begeleiding en zonder veel tijd te verliezen, zouden verbale antwoorden gegeven kunnen worden.

Een mooi voorbeeld van een gelijkaardige interactie is het project 'Dimensions in Testimony' van de \textcite{USCShoahFoundation2020}. Daar kunnen bezoekers vragen stellen aan een overlevende van de Holocaust die op voorhand een volledig interview heeft afgelegd. Dit werd mogelijk gemaakt dankzij het systeem dat erachter zit (Figuur \ref{fig:DiTArchitecture}). Dit systeem is opgebouwd uit volgende componenten: een software voor spraakherkenning (ASR) die de gebruikers verbale vraag in tekst omzet, een Natural Language Classifier (NLC) die op basis van de gegenereerde tekst een antwoord via een audio/video fragment voorziet en een mediaspeler die de fragmenten kan afspelen met tussendoor een inactieve animatie \autocite{Traum2015}.

\begin{figure}[h]
    \centering
    \includegraphics[width=\linewidth*3/4]{TraumDEtAl_DimensionsInTestimony_SystemArchitecture.png}
    \caption{Dimensions in Testimony - Systeem Architectuur \autocite{Traum2015}}
    \label{fig:DiTArchitecture}
\end{figure}

Een ander voorbeeld zijn de home assistenten zoals Google Home, Amazon Echo of Apple HomePod. Zij staan op stand-by tot de gebruiker het triggerwoord zegt, `Hey Google' bijvoorbeeld. Eenmaal geactiveerd luisteren ze naar wat de gebruiker zegt en zetten ze dit aan de hand van een ASR om naar tekst. Daarna haalt het met natuurlijke taalverwerking (NLP) de intentie en sleutelwoorden uit de tekst en genereert daarmee een gepast antwoord. Als laatste zet het gegenereerde tekst om naar spraak.

\section{\IfLanguageName{dutch}{Spraakherkening}{Speech recognition}}%

Net als VR is spraakherkenning de laatste jaren veel vooruitgegaan, dit komt omdat er meer data beschikbaar is, de computers sneller en de algoritmes, genaamd deep neural networks, beter zijn. Deze netwerken worden getraind op grote datasets bestaande uit diverse spraakfragmenten. Hieruit leert het de patronen en kenmerken van menselijke spraak.

Maar perfect zullen ze nooit zijn zegt \textcite{Hessen2020}. Dit is geen verrassing aangezien zelf mensen vaak nog problemen hebben bij het verstaan van een andere. De meest voorkomende fouten van een ASR zijn herkenningsfouten. Deze kunnen ontstaan door slechte kwaliteit van de opname maar ook van de manier waarop woorden worden uitgesproken. De andere fouten ontstaan omdat ASR zijn vocabulaire niet uitgebreid genoeg is. Dit noemen we Out-of-vocabulary-fouten (OOV) \autocite{Hessen2020}.

\section{\IfLanguageName{dutch}{Natuurlijke taalverwerking}{Natural Language Classifier}}%

Naast de ASR hebben we ook de natuurlijke taalverwerking (NLP). In 'Dimension in Testimony' maakten ze hiervoor gebruik van NCPEditor (Figuur \ref{fig:NPCEArchitecture}). De classifier in dit systeem berekent welke antwoorden op de ingegeven tekst kunnen gegeven worden door de taalmodellen van beide te vergelijken en de antwoorden te rangschikken. Zolang de ingegeven tekst een foutmarge lager dan 50\% blijft, zal het antwoord ongewijzigd blijven \autocite{Leuski2010}.

Een andere bekend taalmodel is GPT-3. Dit model ligt aan de basis van de populaire chatbot ChatGPT.

\begin{figure}[h]
    \centering
    \includegraphics[width=\linewidth*3/4]{Traum&Leuski_NPCEditor_System architecture.png}
    \caption{NPCEditor - Systeem Architectuur \autocite{Leuski2010}}
    \label{fig:NPCEArchitecture}
\end{figure}

%Dit hoofdstuk bevat je literatuurstudie. De inhoud gaat verder op de inleiding, maar zal het onderwerp van de bachelorproef *diepgaand* uitspitten. De bedoeling is dat de lezer na lezing van dit hoofdstuk helemaal op de hoogte is van de huidige stand van zaken (state-of-the-art) in het onderzoeksdomein. Iemand die niet vertrouwd is met het onderwerp, weet nu voldoende om de rest van het verhaal te kunnen volgen, zonder dat die er nog andere informatie moet over opzoeken \autocite{Pollefliet2011}.
%
%Je verwijst bij elke bewering die je doet, vakterm die je introduceert, enz.\ naar je bronnen. In \LaTeX{} kan dat met het commando \texttt{$\backslash${textcite\{\}}} of \texttt{$\backslash${autocite\{\}}}. Als argument van het commando geef je de ``sleutel'' van een ``record'' in een bibliografische databank in het Bib\LaTeX{}-formaat (een tekstbestand). Als je expliciet naar de auteur verwijst in de zin, gebruik je \texttt{$\backslash${}textcite\{\}}.
%Soms wil je de auteur niet expliciet vernoemen, dan gebruik je \texttt{$\backslash${}autocite\{\}}. In de volgende paragraaf een voorbeeld van elk.
%
%\textcite{Knuth1998} schreef een van de standaardwerken over sorteer- en zoekalgoritmen. Experten zijn het erover eens dat cloud computing een interessante opportuniteit vormen, zowel voor gebruikers als voor dienstverleners op vlak van informatietechnologie~\autocite{Creeger2009}.
%
%\lipsum[7-20]
