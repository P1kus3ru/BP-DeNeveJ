%%=============================================================================
%% Inleiding
%%=============================================================================

\chapter{\IfLanguageName{dutch}{Inleiding}{Introduction}}%
\label{ch:inleiding}

In de reclamespot `The Impact Will Be Real` over de Metaverse toont \textcite{Meta2022} hun visie over de rol die Virtual Reality (VR) speelt in de toekomst. Zo laten ze verschillende toepassingen ervan in het onderwijs zien. Jammer genoeg staat onze technologie nog niet zo ver als wat er in de reclame gezien kan worden maar ook nu al vindt VR zich een baan in verschillende opleidingen.

De Hogeschool van Gent maakt ook gebruik van VR om studenten de kans te geven in meer realistische situaties te oefenen. Hiervoor zijn twee verschillende technieken gebruikt. Allereerst heb je het renderen van een omgeving. Dit laat de gebruiker een interactieve wereld van 3D modellen ontdekken. Zo bestaan er drie virtuele kamers waarin de student kan oefenen. De andere manier is aan de hand van een 360° opname die wordt gemaakt aan de hand van een 360° camera. Omdat dit een opname van de werkelijkheid neemt, ziet deze methode er realistischer uit.

Dit is waar er op het probleem wordt gestoten. Aangezien de tweede methode werkt met een opname moet er naar verschillende fragmenten gesprongen worden naargelang het antwoord dat de gebruiker ingeeft. Dit wordt handmatig gedaan door een begeleider. Hierdoor staat de oefening tijdelijk stil wat de echtheid van de situatie weghaalt. Daarom wordt in deze bachelorproef toegepaste informatica onderzocht hoe het overschakelen anders kan aangepakt worden zodat het voor de gebruiker realistischer aanvoelt. Hiervoor kijken we richting Artificiële Intelligentie (AI).

Het afgelopen jaar is de populariteit van AI enorm gestegen. Met text-to-image models zoals DALL-E 2, Imagen en Stable diffusion die een gegeven tekst prompt kunnen omzetten in afbeeldingen,


%De inleiding moet de lezer net genoeg informatie verschaffen om het onderwerp te begrijpen en in te zien waarom de onderzoeksvraag de moeite waard is om te onderzoeken. In de inleiding ga je literatuurverwijzingen beperken, zodat de tekst vlot leesbaar blijft. Je kan de inleiding verder onderverdelen in secties als dit de tekst verduidelijkt. Zaken die aan bod kunnen komen in de inleiding~\autocite{Pollefliet2011}:
%
%\begin{itemize}
%  \item context, achtergrond
%  \item afbakenen van het onderwerp
%  \item verantwoording van het onderwerp, methodologie
%  \item probleemstelling
%  \item onderzoeksdoelstelling
%  \item onderzoeksvraag
%  \item \ldots
%\end{itemize}

\section{\IfLanguageName{dutch}{Probleemstelling}{Problem Statement}}%
\label{sec:probleemstelling}

Wanneer de stotterpatiënt antwoordt op de gestelde vraag wordt er niet automatisch overgeschakeld naar een volgend fragment. Er moet namelijk handmatig geklikt worden op het gewenste hoofdstuk door iemand die de oefening kent. Dit zorgt dat er steeds een begeleider aanwezig moet zijn. Ook zal de dynamiek van de oefening verbroken worden omdat het pas verder kan gaan wanneer de begeleider het juiste fragment vindt.

%Uit je probleemstelling moet duidelijk zijn dat je onderzoek een meerwaarde heeft voor een concrete doelgroep. De doelgroep moet goed gedefinieerd en afgelijnd zijn. Doelgroepen als ``bedrijven,'' ``KMO's'', systeembeheerders, enz.~zijn nog te vaag. Als je een lijstje kan maken van de personen/organisaties die een meerwaarde zullen vinden in deze bachelorproef (dit is eigenlijk je steekproefkader), dan is dat een indicatie dat de doelgroep goed gedefinieerd is. Dit kan een enkel bedrijf zijn of zelfs één persoon (je co-promotor/opdrachtgever).

\section{\IfLanguageName{dutch}{Onderzoeksvraag}{Research question}}%
\label{sec:onderzoeksvraag}

Om dit probleem te verhelpen wordt hiervoor gekeken hoe AI ons kan te hulp schieten om stotterpatiënten een meer interactieve ervaring geven. Om dit te realiseren zullen de volgende vragen beantwoord worden:
\begin{itemize}
    \item Hoe gaat de applicatie registreren wat er werd gezegd?
    \item Wat kan er mis lopen bij het registreren van spraak?
    \item Hoe kiest de applicatie een volgend fragment op basis van de gegenereerde tekst?
\end{itemize}

%Wees zo concreet mogelijk bij het formuleren van je onderzoeksvraag. Een onderzoeksvraag is trouwens iets waar nog niemand op dit moment een antwoord heeft (voor zover je kan nagaan). Het opzoeken van bestaande informatie (bv. ``welke tools bestaan er voor deze toepassing?'') is dus geen onderzoeksvraag. Je kan de onderzoeksvraag verder specifiëren in deelvragen. Bv.~als je onderzoek gaat over performantiemetingen, dan

\section{\IfLanguageName{dutch}{Onderzoeksdoelstelling}{Research objective}}%
\label{sec:onderzoeksdoelstelling}

In deze bachelorproef word een proof of concept opgesteld om te kijken hoe zo een AI-applicatie te werk zou kunnen gaan. De bedoeling van deze applicatie is eerst en vooral registreert wanneer de gebruiker aan het antwoorden is. Vervolgens transcribeert het wat er gezegd wordt. Als laatste gaat het op basis van de context een gepast fragment zoeken om verder mee te gaan.

%Wat is het beoogde resultaat van je bachelorproef? Wat zijn de criteria voor succes? Beschrijf die zo concreet mogelijk. Gaat het bv.\ om een proof-of-concept, een prototype, een verslag met aanbevelingen, een vergelijkende studie, enz.

\section{\IfLanguageName{dutch}{Opzet van deze bachelorproef}{Structure of this bachelor thesis}}%
\label{sec:opzet-bachelorproef}

% Het is gebruikelijk aan het einde van de inleiding een overzicht te
% geven van de opbouw van de rest van de tekst. Deze sectie bevat al een aanzet
% die je kan aanvullen/aanpassen in functie van je eigen tekst.

De rest van deze bachelorproef is als volgt opgebouwd:

In Hoofdstuk~\ref{ch:stand-van-zaken} wordt een overzicht gegeven van de stand van zaken binnen het onderzoeksdomein, op basis van een literatuurstudie.

In Hoofdstuk~\ref{ch:methodologie} wordt de methodologie toegelicht en worden de gebruikte onderzoekstechnieken besproken om een antwoord te kunnen formuleren op de onderzoeksvragen.

% TODO: Vul hier aan voor je eigen hoofstukken, één of twee zinnen per hoofdstuk

In Hoofdstuk~\ref{ch:conclusie}, tenslotte, wordt de conclusie gegeven en een antwoord geformuleerd op de onderzoeksvragen. Daarbij wordt ook een aanzet gegeven voor toekomstig onderzoek binnen dit domein.