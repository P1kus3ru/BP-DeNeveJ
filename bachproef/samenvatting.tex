%%=============================================================================
%% Samenvatting
%%=============================================================================

% TODO: De "abstract" of samenvatting is een kernachtige (~ 1 blz. voor een
% thesis) synthese van het document.
%
% Een goede abstract biedt een kernachtig antwoord op volgende vragen:
%
% 1. Waarover gaat de bachelorproef?
% 2. Waarom heb je er over geschreven?
% 3. Hoe heb je het onderzoek uitgevoerd?
% 4. Wat waren de resultaten? Wat blijkt uit je onderzoek?
% 5. Wat betekenen je resultaten? Wat is de relevantie voor het werkveld?
%
% Daarom bestaat een abstract uit volgende componenten:
%
% - inleiding + kaderen thema
% - probleemstelling
% - (centrale) onderzoeksvraag
% - onderzoeksdoelstelling
% - methodologie
% - resultaten (beperk tot de belangrijkste, relevant voor de onderzoeksvraag)
% - conclusies, aanbevelingen, beperkingen
%
% LET OP! Een samenvatting is GEEN voorwoord!

%%---------- Nederlandse samenvatting -----------------------------------------
%
% TODO: Als je je bachelorproef in het Engels schrijft, moet je eerst een
% Nederlandse samenvatting invoegen. Haal daarvoor onderstaande code uit
% commentaar.
% Wie zijn bachelorproef in het Nederlands schrijft, kan dit negeren, de inhoud
% wordt niet in het document ingevoegd.

\IfLanguageName{english}{%
\selectlanguage{dutch}
\chapter*{Samenvatting}
%\lipsum[1-4]
\selectlanguage{english}
}{}

%%---------- Samenvatting -----------------------------------------------------
% De samenvatting in de hoofdtaal van het document

\chapter*{\IfLanguageName{dutch}{Samenvatting}{Abstract}}

Het 360° Zorglab van de Hogeschool Gent biedt stotterpatiënten, aan de hand van interactieve video's met Virtual Reality, kansen aan om in realistische situaties met stotteren om te leren gaan. Per scenario bestaat er een filmpje dat in fragmenten is opgedeeld maar om naar het volgende fragment te gaan moet de gebruiker zelf zijn antwoord nog manueel aanduiden. Dit haalt de dynamiek uit de oefening weg. Daarom wordt er in deze bachelorproef informatica gezocht naar een oplossing om dit te verhelpen. Om dit te verwerven wordt eerst een literatuurstudie uitgevoerd. Deze bevat informatie over wat het stotteren is en veroorzaakt, de huidige toestand en geschiedenis van Virtual Reality en de stand van Artificiële Intelligentie met een uitgebreidere analyse van spraakherkenning en natuurlijke taalverwerking. Als tweede wordt er lijst opgesteld van bestaande spraakherkennings- en natuurlijke taalverwerkingssoftware. Vervolgens wordt er een proof of concept opgesteld en aan de hand daarvan verschillende spraakherkenningsmodellen vergeleken op spraakfragmenten met en zonder stottermomenten. Uit de resultaten valt op dat Whisper's transcripties de verstaanbaarheid van gesproken fragmenten behouden in zowel het kleine als medium model, ondanks enkele herkenningsfouten. Het kleine model verwerkt 3 à 4 keer sneller, maar met meer fouten. Google ASR is over het algemeen trager dan Whisper, met nauwkeurigheidsinvloeden van stottermomenten en gebrek aan leestekens. Microsoft Azure speech services zijn sneller dan Google, maar minder nauwkeurig bij stotterende spraak door herhaalde woorden. Over het algemeen blijkt het medium Whisper-model het meest geschikt, ondanks langere verwerkingstijd dan Azure en kleine Whisper-model, vanwege consistente nauwkeurigheid. De betaalde services als Google en Microsoft presteerden minder goed dan verwacht bij stotterfragmenten. Het onderzoek biedt waarde voor het 360° Zorglab en kan dienen als basis voor verdere toepassingen en onderzoek naar spraakherkenning bij stotteren.








