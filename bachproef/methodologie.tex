%%=============================================================================
%% Methodologie
%%=============================================================================

\chapter{\IfLanguageName{dutch}{Methodologie}{Methodology}}%
\label{ch:methodologie}

%% TODO: Hoe ben je te werk gegaan? Verdeel je onderzoek in grote fasen, en
%% licht in elke fase toe welke stappen je gevolgd hebt. Verantwoord waarom je
%% op deze manier te werk gegaan bent. Je moet kunnen aantonen dat je de best
%% mogelijke manier toegepast hebt om een antwoord te vinden op de
%% onderzoeksvraag.

%\lipsum[21-25]

Dit onderzoek start met een literatuurstudie (\ref{chap:State of the art}). Vervolgens wordt er een requirementsanalyse gehouden waar gekeken wordt wat er van de oplossing verwacht wordt (\ref{sect:Requirements analysis}). Na de requirements vast te leggen zullen verschillende spraak naar tekst software (\ref{sect:Speech to text}) en taalverwerkingsmodellen (\ref{sect:Language models}) getest worden. Als laatste zal er een proof-of-concept opgesteld worden.

\section{\IfLanguageName{dutch}{Requirementsanalyse}{Requirements analysis}} \label{sect:Requirements analysis}%

dfdsf

\paragraph{\IfLanguageName{dutch}{Functionele requirements}{Functional requirements}}

dfs

\paragraph{\IfLanguageName{dutch}{Niet-functionele requirements}{Non-functional requirements}}

dfs

\subsection{\IfLanguageName{dutch}{MoSCoW}{MoSCoW}}%

Aan de hand van de MoSCoW methode bepalen we het belang van verschillende functionaliteiten.

\begin{enumerate}
    \item Must have:
    \begin{itemize}
        \item Het detecteren van wanneer de gebruiker spreekt.
        \item Het transcriberen van wat er gezegd wordt.
    \end{itemize}
    \item Should have:
    \begin{itemize}
        \item Aan de hand van de gegenereerde tekst het volgend fragment bepalen.
    \end{itemize}
    \item Could have:
    \begin{itemize}
        \item Een flexibel systeem met verschillende scenario’s creëren
    \end{itemize}
\end{enumerate}

\subsection{\IfLanguageName{dutch}{Use Cases}{Use Cases}}%



\section{\IfLanguageName{dutch}{Spraak naar tekst}{Speech to text}} \label{sect:Speech to text}%

dfsdfsdf


\section{\IfLanguageName{dutch}{Taalmodellen}{Language models}} \label{sect:Language models}%

dfsdfsd


\section{\IfLanguageName{dutch}{Proof-of-concept}{Proof-of-concept}} \label{sect:Proof-of-concept}%

dfsdfsd